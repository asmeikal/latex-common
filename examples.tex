\documentclass[draft]{article}

\usepackage[T1]{fontenc}
\usepackage[italian]{babel}
\usepackage{hyperref}
\usepackage{nameref}
\usepackage{fullpage}
\usepackage{parskip}

\usepackage{mlbasemath}
\usepackage{mlalgebra}
\usepackage{mlalgoritmi}
\usepackage{mlcombinatoria}
\usepackage{mlcomplessita}
\usepackage{mlprobabilita}

\renewcommand{\arraystretch}{1.5}

\begin{document}

Nel prendere appunti in \LaTeX{} cerco di rispecchiare la semantica di quel che sto scrivendo.
Per questo preferisco scrivere \verb|\implies| a scrivere \verb|\Rightarrow|.

\section{mlbasemath}

Questo package fornisce alcuni operatori di base.
Tutti gli altri package dovrebbero includerlo.

\begin{center}
\begin{tabular}{ | p{7cm} | p{4cm} | p{4cm} | }
	\hline
	Descrizione & Comando & Risultato \\
	\hline
	minimo comune multiplo, massimo comun divisore & \verb|\mcm(a,b)|, \verb|\mcd(a,b)| & $\mcm(a,b)$, $\mcd(a,b)$ \\ \hline
	segno & \verb|\sgn(x)| & $\sgn(x)$ \\ \hline
	identit\`a & \verb|\id| & $\id$ \\ \hline
	insieme immagine di una funzione & \verb|\image{#1}| & $\image{f}$ \\ \hline
	insieme delle parti & \verb|\parts{#1}| & $\parts{A}$ \\ \hline
	valore assoluto e norma & \verb|\abs{#1}|, \verb|\norm{#1}| & $\abs{x}$, $\norm{x}$ \\ \hline
	intero inferiore e superiore & \verb|\intinf{#1}|, \verb|\intsup{#1}| & $\intinf{x}$, $\intsup{x}$ \\ \hline
	operatore ``divide'' & \verb|\divides| & $a \divides b$ \\ \hline
	operatore binario ``per definizione'' & \verb|\definition| & $n! \definition n (n-1) \dots 1$ \\ \hline
	intervalli aperti e chiusi & \verb|\ccint{#1}{#2}|, \verb|\coint{#1}{#2}|, \verb|\ocint{#1}{#2}|, \verb|\ooint{#1}{#2}| & $\ccint{-1}{1}$, $\coint{1}{\infty}$, $\ocint{-\infty}{\sqrt{2}}$, $\ooint{\pi}{2\pi}$\\ \hline
	insiemi numerici & \verb|\naturals|, \verb|\integers|, \verb|\rationals|, \verb|\reals|, \verb|\complexes| & $\naturals$, $\integers$, $\rationals$, $\reals$, $\complexes$ \\ \hline
	insieme degli interi da 1 a $n$ & \verb|\ints{#1}| & $\ints{n}$ \\ \hline
	pezzi di codice & \verb|\code{#1}| & \code{int main(void)} \\ \hline
\end{tabular}
\end{center}

Il package contiene anche discutibili scelte di stile:
\begin{itemize}
	\item I quantificatori sono ridefiniti da $\oldforall x$ a $\forall x$ e da $\oldexists x$ a $\exists x$: mi piaceva che avessero pi\`u spazio intorno.
	\`E una scelta fatta quando un anno fa, che non mi convince pi\`u molto.
	\item L'implicazione e la doppia implicazione usano frecce pi\`u corte: $\iff$ e $\implies$.
\end{itemize}

Ci sono infine diversi teoremi.
\begin{center}
\begin{tabular}{| c | c |}
\hline
Nome & Descrizione \\
\hline
\code{axiom} & assiona \\ \hline
\code{theorem} & teorema \\ \hline
\code{lem} & lemma \\ \hline
\code{prop} & proposizione \\ \hline
\code{cor} & corollario \\ \hline
\code{defn} & definizione \\ \hline
\code{esercizio} & esercizio \\ \hline
\code{remark} & ``cosa ribadita'' (?) \\ \hline
\code{oss} & osservazione \\ \hline
\code{caso} & caso \\ \hline
\code{fact} & fatto \\ \hline
\code{claim} & affermazione \\ \hline
\end{tabular}
\end{center}

\section{mlalgebra}

Giusto due comandi.

\begin{center}
\begin{tabular}{ | p{7cm} | p{4cm} | p{4cm} | }
	\hline
	Descrizione & Comando & Risultato \\
	\hline
	insieme dei sottogruppi & \verb|\subgroupset| & $\subgroupset$ \\ \hline
	insieme delle soluzioni & \verb|\solutions| & $\solutions$ \\ \hline
\end{tabular}
\end{center}

\section{mlalgoritmi}

Fornisce gli ambienti \code{algorithm} e \code{algorithmic}.
Informazioni qui: \url{https://en.wikibooks.org/wiki/LaTeX/Algorithms#The_algorithm_environment}.

\begin{algorithm}
\begin{algorithmic}[1]
\Require Un certo input
\Ensure Un certo output
\State Fai qualcosa
\State \Return un risultato
\end{algorithmic}
\end{algorithm}
\begin{verbatim}
\begin{algorithm}
\begin{algorithmic}[1]
\Require Un certo input
\Ensure Un certo output
\State Fai qualcosa
\State \Return un risultato
\end{algorithmic}
\end{algorithm}
\end{verbatim}

\section{mlcombinatoria}

Anche qui, poca roba per la combinatoria.

\begin{tabular}{ | p{7cm} | p{4cm} | p{4cm} | }
	\hline
	Descrizione & Comando & Risultato \\
	\hline
	parit\`a di un elemento & \verb|\parity{#1}| & $\parity{2}$ \\ \hline
	numero di clique in un grafo & \verb|\clique{#1}| & $\clique{G}$ \\ \hline
	grado di un nodo (\verb|#2|) in un grafo (opzionale, \verb|#1|) & \verb|\degree[#1]{#2}| & $\degree[G]{x}$ \\ \hline
\end{tabular}


\section{mlcomplessita}

Un po' di cose per automi/complessit\`a/calcolabilit\`a.

\begin{center}
\begin{tabular}{ | p{7cm} | p{4cm} | p{4cm} | }
	\hline
	Descrizione & Comando & Risultato \\
	\hline
	insieme dei linguaggi di un modello di calcolo & \verb|\langs{#1}| & $\langs{\code{DFA}}$ \\ \hline
	linguaggio di una macchina/automa & \verb|\lang{#1}| & $\lang{M}$ \\ \hline
	\emph{or} nelle grammatiche & \verb|\langor| & $S \to A \langor B$ \\ \hline
	stella di Kleene & \verb|\kleenestar| & $\{0,1\}^{\kleenestar}$ \\ \hline
	strinva vuota & \verb|\emptystring| & $\emptystring$ \\ \hline
	simbolo di fine stack & \verb|\stackend| & $\stackend$ \\ \hline
	simbolo di relazione ``porta a'' fra configurazioni & \verb|\leadsto| & $\leadsto$ \\ \hline
	``O grande'' & \verb|\bigo{#1}| & $\bigo{n^2}$ \\ \hline
	classi di problemi & \verb|\probp|, \verb|\probnp|, \verb|\probnpc|, \verb|\probexp| & $\probp$, $\probnp$, $\probnpc$, $\probexp$ \\ \hline
	\emph{or} e \emph{and} grandi (prefissi) & \verb|\Lor|, \verb|\Land| & $\Lor$, $\Land$ \\ \hline
	simbolo di cella vuota & \verb|\emptysymbol| & $\emptysymbol$ \\ \hline
\end{tabular}
\end{center}

\section{mlprobabilita}

Un po' di cose per probabilit\`a.

\begin{center}
\begin{tabular}{ | p{7cm} | p{4cm} | p{4cm} | }
	\hline
	Descrizione & Comando & Risultato \\
	\hline
	varianza & \verb|\var| & $\var$ \\ \hline
	probabilit\`a (il parametro opzionale cambia il nome della funzione di probabilit\`a) & \verb|\prob[#1]{#2}| & $\prob{E}$ \\ \hline
	probabilit\`a di \verb|#2| condizionata \verb|#3| (il parametro opzionale cambia il nome della funzione di probabilit\`a) & \verb|\probcond[#1]{#2}{#3}| & $\probcond{E}{F}$ \\ \hline
	valore atteso di una variabile aleatoria & \verb|\expect{#1}| & $\expect{X}$ \\ \hline
	?? (il parametro opzionale cambia il nome della variabile aleatoria) & \verb|\mass[#1]{#2}| & $\mass{E}$ \\ \hline
	distribuzione (il parametro opzionale cambia il nome della variabile aleatoria) & \verb|\distr[#1]{#2}| & $\distr{E}$ \\ \hline
\end{tabular}
\end{center}

\end{document}

